\chapter{Introduction}

\paragraph{}

Le moustique tigre \esp{Aedes albopictus} est un insecte qui suscite un intérêt croissant depuis ces dix dernières années.
En effet, cet insecte est vecteur potentiel de 26 arbovirus\footnote{[Ref. néc.]}, parmi lesquels figurent les désormais célèbres DENV (maladie de la Dengue) et CHKV (Maladie du Chikungunya).
En plus de l'aspect vectoriel de cet insecte de l'ordre des diptères et de la famille des \textit{Culicidæ}, celui-ci présente des capacités adaptatives lui permettant de coloniser des climats très différents\footnote{[Ref. néc.]}, capacités qui, combinées avec le trafic commercial humain mondial\footnote{[Ref. néc.]}, fait du moustique tigre le moustique le plus invasif de ces dix dernières années.

Plusieurs facteurs peuvent expliquer un tel succès adaptatif. Parmi celles-ci, on peut citer la capacité de la plupart des lignées ce ce moustique à rentrer en diapause sous l'état d'\oe{}uf, ou sa résistance importante à divers polluants\footnote{[Réf. néc.]}.

Chez d'autres modèles insectes, il a été mis de nombreuses fois en évidence l'importance de la microflore symbiotique, pouvant interférer avec de nombreuses fonctions vitales de leur hôte.
% Exemples
Nous pouvons citer comme exemple les rôles biens connus de l'endosymbiote \esp{Wolbachia}, présent dans de nombreux groupes insectes, dans les fonctions de reproduction de leur hôte, et, comme décrit par [ref], dans la compétence vectorielle du moustique [ref].
D'autres études montrent que les bactéries cultivables peuvent jouer des rôles dans des fonctions importantes de leur hôte, comme \esp{Lactobacillus plantarum} dans les préférences sexuelles de la drosophile [Sharon2010].
Plus spécifiquement pour les fonctions ayant un rôle dans l'adaptation et donc l'invasivité des insectes, un rôle prédominant de la bactérie symbiotique \esp{Buchnera} dans la résistance au stress thermique a été mis en évidence, chez l'hôte \esp{A. pisum}, un puceron ravageur de cultures.
Ces interférences, en apportant des avantages évolutifs à leur hôte, permettent la sélection d'holobiontes\footnote{Déf. holobionte} pouvant potentiellement être plus adaptés à des stress, notamment climatiques.

Chez \esp{Aedes albopictus}, la présence de la bactérie endosymbiotique \esp{Wolbachia} ainsi que son interférence dans de nombreuses fonctions de leur hôte, comme la reproduction ou la compétence vectorielle, est déjà bien établi.
En parallèle de cela, il a été récemment mis en évidence la forte prévalence de certaines espèces bactériennes associées à \esp{Aedes albopictus}, comme \esp{Asaia}, \esp{Acinetobacter} et \esp{Pantoea}, toutes les trois cultivables.
Au vu de cette forte prévalence, densité au sein du moustique, et du caractère vertical de la transmission de certains d'entre eux, ces symbiotes peuvent être considérés comme de bons candidats pour tester l'hypothèse d'un rôle du symbiome dans les capacités d'adaptation du moustique, notamment au stress thermique, élément clé lors d'une remontée vers le Nord comme celle du moustique tigre ces dernières années.

Le but de mon stage de master est de mettre en place les outils nécessaire à l'étude des potentielles interférences des communautés symbiotiques cultivables, plus spécifiquement d'\esp{Acinetobacter calcoaceticus} dans différents traits d'histoire de vie du moustique. 
Cette mise en place consiste en l'obtention d'une bactérie marquée permettant son suivi \textit{in situ}, en la conception des différentes techniques d'élevage, de traitement et de suivi des populations de moustiques en insectarium, et en le suivi par biologie moléculaire et microscopie des statuts d'infection de ces moustiques.
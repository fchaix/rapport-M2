\chapter{Matériel et méthodes}

\section{Matériel biologique et génétique}

\subsection{Moustiques \textit{Aedes albopictus}}
\label{par:moustiques}
La population de moustiques \esp{Aedes albopictus} provient d'une lignée d'élevage de laboratoire, hébergée par le Centre de Recherche et de Veille sur les Maladies Emergentes dans l'Océan Indien (CRVOI), à l’île de la Réunion.
Les moustiques originels ont été capturés en 2012 dans trois localités distinctes de l’Île de la Réunion (Saint Benoît, Sainte Suzanne et Saint Denis) afin de minimiser les problèmes de consanguinité.

\subsection{Isolat bactérien}
%Le choix de l'espèce \esp{Acinetobacter calcoaceticus} a été motivé par sa forte prévalence mise en évidence lors d'études précédentes [Publi Guillaume], ainsi que par sa cultivabilité et les capacités de transformation décrites dans de nombreuses souches de ce genre bactérien [Ref.].
La souche d'\esp{Acinetobacter calcoaceticus} a été obtenue suite à l’isolement sur milieu riche de culture d’un broyat de moustique Ae. albopictus mâle. Ce dernier était issu d’une population de laboratoire dont la population originelle avait été capturée sur l’Île de la Réunion en 2006 (\textcite{zouache2009})
Le séquençage complet du gène codant l’ARNr 16s a montré 99\% d’identité avec l’espèce A. calcoaceticus. L’isolat a été déposé dans GenBank sous le numéro d’accession FJ688379 et l’identifiant KZ-OAlM.
% Pourquoi la souche D ?

\subsection{Plasmides}

Deux plasmides ont été utilisés lors des essais de transformation de l’isolat d’A. calcoaceticus en vue de son suivi \textit{in insecta} sont les suivants~:
\begin{itemize}
\item \textbf{Le plasmide pHM2-GFP} est non mobilisable (absence de gènes de l'opéron \textit{tra}), et possède une origine de réplication (\textit{Rep}) généraliste lui permettant de se répliquer dans de nombreux groupes bactériens.
Outre le gène marqueur \textit{GFP} (Green Fluorescent Protein), cloné sur le site NotI, un autre marqueur~: l'opéron lactose, induisant une coloration bleue des colonies en présence de X-Gal, et un gène de résistance à la Kanamycine.
Le plasmide a été fourni par le docteur Guido Favio du Département de Médecine Expérimentale et de Santé Publique de Camerino en Italie. La carte du plasmide est présentée en Figure 1.\cite{favia2007}

\item \textbf{Le plasmide pXDC116} contient, outre le gène marqueur de la \textit{GFP} et l'opéron lactose, un gène de résistance à la kanamycine et au chloramphenicol. Ce plasmide est néanmoins mobilisable (présence d'une origine de transfert \textit{oriT}, et des gènes \textit{mobC} et \textit{mobB}).
Ce plasmide a été fourni par le docteur Xavier Charpentier du laboratoire de Microbiologie, Adaptation et Pathogénié de Lyon et a déjà été utilisé avec succès pour transformer des bactéries du genre Acinetobacter. La carte du plasmide est présentée en Figure 2.

\end{itemize}

% \begin{figure}[h]
%    \includegraphics[width=\textwidth]{images/pXDC116.png}
%    \caption{plop}
% \end{figure}

\section{Mise en place de l'élevage d'\esp{Aedes albopictus} et mesures des traits d'histoire de vie (THV)}

\subsection{Mise en place de l'élevage}

L’élevage d'Ae. albopictus a été réalisé en insectarium de niveau de confinement I2 en partenariat avec la Rovaltain Research Company situé à Alixan, à proximité de Valence. Pour débuter cet élevage, des œufs provenant d’une population d’élevage du CRVOI (voir paragraphe 2.1.1) de la génération G31 ont été mis en bassine contenant de l'eau déchlorée, à une température ambiante de 25°C (Figure). 
Après éclosion des œufs, les larves ont été nourries avec un mélange composé de 3/4 de nourriture pour poisson d'aquarium (TetraMin ®) et 1/4 de levure de bière (Biover®).
À l'émergence, les adultes ont été maintenus à 25°C ±1°C, une humidité relative de 80\,\% (±10\,\%) et une photopériode de XXX, avec une solution nutritive stérilisée d'eau glucosée à 8\,\%. Les femelles adultes ont ensuite été nourries une fois par semaine sur du sang de lapin (donner détails de la composition et l’origine) au moyen de nourrisseurs artificiels Hemotek (figure).
Les œufs pondus ont été récupérés régulièrement puis mis à éclore.

\subsection{Conception et mise en \oe{}uvre des expériences de mesure de traits d'histoire de vie}

En se basant sur les conseils de l'entomologiste \textsc{Raharimalala} Fara Nantenaina et les documents délivrés par l'ECDC\footnote{European Center for Disease Prevention and Control} \cite{ecdc}, nous avons planifié de mesurer divers traits d'histoire de vie comme suit :

\paragraph{Survie des adultes.} Ce trait est mesuré sur des moustiques du même âge (±\,2 jours), isolés dans des petites cages (17.5\,×\,17,5\,×\,17.5\,cm). La date et le sexe des morts seront notés au fur et à mesure du déroulement de l'expérience, et stockés dans une base de données pour des analyses statistiques ultérieures.
En raison de la forte mortalité subie par les moustiques ayant subi le traitement antibiotique (voir section 2.4), et du faible nmbre d'individus adultes qui en résulte, il n'a pas été possible d'isoler des cages de 30 individus de ces conditions.
Ce suivi n'a pu être effectué que sur les individus témoins.

\paragraph{Taille de la ponte.} Ce trait est mesuré en isolant des femelles juste après leur premier gorgement, dans des tubes Falcon de 50\,ml préparés pour la ponte (figure 3A). Le nombre d'\oe{}ufs par femelle est compté 4 jours après l'isolement.
Un autre trait, la \textbf{fécondité}, est mesuré à partir de ces \oe{}fs. Il s'agit du rapport [Nombre d'\oe{}ufs par femelle/Nombre de larves], déterminé suite au dénombrement des larves une semaine après le comptage des \oe{}ufs.

\paragraph{Durée du cycle gonotrophique.} Il s'agit du temps séparant le premier gorgement et la première ponte d'une femelle. Ce trait est mesuré sur les femelles isolées du paragraphe précédent. La présence ou non d'\oe{}ufs est vérifiée chaque jour qui suit le gorgement.

\section{Obtention d'une souche d'\esp{Acinetobacter calcoaceticus} marquée à la GFP}

\subsection{Électrotransformation}



\subsection{Vérifications post-transformation}

\paragraph{PCRs}
\paragraph{Profil plasmidique}
\paragraph{Repiquages successifs}
\paragraph{Observation au microscope à épifluorescence}

\section{Traitement antibiotique des moustiques}

\subsection{Antibiogramme de la souche A. calcoaceticus et modalités de traitement antibiotique}

\subsection{Application du traitement}

\section{Infection des moustique par \esp{Acinetobacter calcoaceticus}-GFP}

\section{Suivi post-infection de \esp{Acinetobacter calcoaceticus}}

[PCR nichée pA/pH, Acineto]
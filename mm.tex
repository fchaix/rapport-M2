\chapter{Matériel et méthodes}

\section{Matériel biologique et génétique}

\paragraph{Moustiques.}
\label{par:moustiques}
La population de moustiques \esp{Aedes albopictus} qui est utilisée dans cette étude est issue d'une lignée d'élevage de laboratoire, provenant initialement de l'île de la Réunion\footnote{La lignée est composée de moustques issus de captures dans trois localités distinctes de l'Île de la Réunion (Saint Benoît, Sainte Suzanne et Saint Denis), afin de minimiser les problèmes de consanguinité.}.
Le stage débute avec la génération 31 de cette ligée de moustiques, comptée depuis sa capture dans le milieu naturel.

\paragraph{Souches bactériennes.}
%Le choix de l'espèce \esp{Acinetobacter calcoaceticus} a été motivé par sa forte prévalence mise en évidence lors d'études précédentes [Publi Guillaume], ainsi que par sa cultivabilité et les capacités de transformation décrites dans de nombreuses souches de ce genre bactérien [Ref.].
La souche bactérienne d'\esp{Acinetobacter calcoaceticus} choisie comme modèle pour ces expériences est issue d'un isolement de la flore cultivable d'un moustique capturé à L'Île de la Réunion, en 2006, puis élevé en laboratoire sur 5 générations.
Il s'agit de la souche identifiée sur GenBank sous le nom de KZ-OAlM, publiée en 2009\cite{zouache2009}.
% Pourquoi la souche D ?

\paragraph{Plasmides}

Les plasmides utilisés afin de marquer la souche et la suivre \textit{in insecta} sont les suivants~:
\begin{itemize}
\item \textbf{pHM2-GFP} (voir figure 1) est un plasmide non mobilisable (absence de gènes de l'opéron \textit{tra}), avec une origine de réplication (\textit{Rep}) généraliste lui permettant de se répliquer dans de nombreux groupes bactériens. Il contient, outre le gène marqueur \textit{GFP} (cloné sur le site NotI, voir figure 1), un autre marqueur, l'opéron lactose, permettant une coloration bleue des colonies en présence de X-Gal, et un gène de résistance à la Kanamycine.
\cite{favia2007}
\item \textbf{pXDC116} (voir figure 2) est un plasmide fourni initialement par le laboratoire MAP en qualité de témoin positif de la transformation. EN effet, il a déjà été utilisé dans des transformations sur des bactéries de genre Acinetobacter (ref).
Ce plasmide contient, outre la \textit{GFP} et l'opéron lactose, un gène de résistance à la kanamycine et au chloramphenicol. Ce plasmide est néenmoins mobilisable (présence d'une origine de transfert \textit{oriT}, et des gènes \textit{mobC} et \textit{mobB}).
\end{itemize}

% \begin{figure}[h]
%    \includegraphics[width=\textwidth]{images/pXDC116.png}
%    \caption{plop}
% \end{figure}

\section{Mise en place de l'élevage d'\esp{Aedes albopictus} et ajustement des mesures de THV}

\subsection{Mise en place de l'élevage}

L'élevage et les mesures sur moustique vivant sont effectués en insectarium de niveau de confinement n°2 (conformément à la Directive 2008/61/CE), grâce à un partenariat avec la Rovaltain Research Company (anciennement Pôle Écotox), à Valence.

%[Description du démarrage de l'élevage a partir des \oe{}ufs désséchés]
Les moustiques sont élevés dans l'insectarium de Rovaltain suivant un protocole synthétisant celui utilisé à l'Institut Pasteur, avec les indications fournies par le Dr. \textsc{Raharimalala} Fara Nantenaina, entomologiste à [labo fara à compléter].
L'élevage de la lignée commence par la génération G31, et sera élevé dans des conditions optimales pour \textit{Aedes albopictus} (25°C (±1), 80\,\% d'humidité relative (±10\,\%), nourris en routine avec une silution de saccharose à 8\,\%, gorgés chaque semaine avec du sang de lapin grâce au dispositif Hémotek©), jusqu'à la génération G32.

La génération G33 est celle qui sera utilisée pour les expériences d'infection bactérienne (voir partie 2.5).

\subsection{Conception et mise en \oe{}uvre des expériences de mesure de traits d'histoire de vie}

En se basant sur les conseils de l'entomologiste \textsc{Raharimalala} Fara Nantenaina et les documents délivrés par l'ECDC\footnote{European Center for Disease Prevention and Control} \cite{ecdc}, nous avons planifié de mesurer divers traits d'histoire de vie comme suit :

\paragraph{Survie des adultes.} Ce trait est mesuré sur des moustiques du même âge (±\,2 jours), isolés dans des petites cages (17.5\,×\,17,5\,×\,17.5\,cm). La date et le sexe des morts seront notés au fur et à mesure du déroulement de l'expérience, et stockés dans une base de données pour des analyses statistiques ultérieures.

\paragraph{Taille de la ponte.} Ce trait est mesuré en isolant des femelles juste après leur premier gorgement, dans des tubes Falcon de 50\,ml préparés pour la ponte (figure 3A). Le nombre d'\oe{}ufs par femelle est compté 4 jours après l'isolement.
Un autre trait, la \textbf{fécondité}, est mesuré à partir de ces \oe{}fs. Il s'agit du rapport [Nombre d'\oe{}ufs par femelle/Nombre de larves], déterminé suite au dénombrement des larves une semaine après le comptage des \oe{}ufs.

\paragraph{Durée du cycle gonotrophique.} Il s'agit du temps séparant le premier gorgement et la première ponte d'une femelle. Ce trait est mesuré sur les femelles isolées du paragraphe précédent. La présence ou non d'\oe{}ufs est vérifiée chaque jour qui suit le gorgement.

\section{Obtention d'une souche d'\esp{Acinetobacter calcoaceticus} marquée à la GFP}

\subsection{Électrotransformation}

\subsection{Vérifications post-transformation}

\paragraph{PCRs}
\paragraph{Profil plasmidique}
\paragraph{Repiquages successifs}
\paragraph{Observation au microscope à épifluorescence}

\section{Traitement antibiotique des moustiques}

\subsection{Antibiogramme de la souche A. calcoaceticus et modalités de traitement antibiotique}

\subsection{Application du traitement}

\section{Infection des moustique par \esp{Acinetobacter calcoaceticus}-GFP}

\section{Suivi post-infection de \esp{Acinetobacter calcoaceticus}}

[PCR nichée pA/pH, Acineto]
\chapter{Matériel et méthodes}

\section{Matériel biologique et génétique}

\subsection{Moustiques \textit{Aedes albopictus}}
\label{par:moustiques}
La population de moustiques \esp{Aedes albopictus} provient d'une lignée d'élevage de laboratoire, hébergée par le Centre de Recherche et de Veille sur les Maladies Emergentes dans l'Océan Indien (CRVOI), à l’île de la Réunion.
Les moustiques originels ont été capturés en 2012 dans trois localités distinctes de l’Île de la Réunion (Saint Benoît, Sainte Suzanne et Saint Denis) afin de minimiser les problèmes de consanguinité.

\subsection{Isolat bactérien}
%Le choix de l'espèce \esp{Acinetobacter calcoaceticus} a été motivé par sa forte prévalence mise en évidence lors d'études précédentes [Publi Guillaume], ainsi que par sa cultivabilité et les capacités de transformation décrites dans de nombreuses souches de ce genre bactérien [Ref.].
La souche d'\esp{Acinetobacter calcoaceticus} a été obtenue suite à l’isolement sur milieu riche de culture d’un broyat de moustique Ae. albopictus mâle. Ce dernier était issu d’une population de laboratoire dont la population originelle avait été capturée sur l’Île de la Réunion en 2006 (\textcite{zouache2009})
Le séquençage complet du gène codant l’ARNr 16s a montré 99\% d’identité avec l’espèce A. calcoaceticus. L’isolat a été déposé dans GenBank sous le numéro d’accession FJ688379 et l’identifiant KZ-OAlM.
% Pourquoi la souche D ?

\subsection{Plasmides}

Deux plasmides ont été utilisés lors des essais de transformation de l’isolat d’A. calcoaceticus en vue de son suivi \textit{in insecta} sont les suivants~:
\begin{itemize}
\item \textbf{Le plasmide pHM2-GFP} est non mobilisable (absence de gènes de l'opéron \textit{tra}), et possède une origine de réplication (\textit{Rep}) généraliste lui permettant de se répliquer dans de nombreux groupes bactériens.
Outre le gène marqueur \textit{GFP} (Green Fluorescent Protein), cloné sur le site NotI, un autre marqueur~: l'opéron lactose, induisant une coloration bleue des colonies en présence de X-Gal, et un gène de résistance à la Kanamycine.
Le plasmide a été fourni par le docteur Guido Favio du Département de Médecine Expérimentale et de Santé Publique de Camerino en Italie. La carte du plasmide est présentée en Figure 1.\cite{favia2007}

\item \textbf{Le plasmide pXDC116} contient, outre le gène marqueur de la \textit{GFP} et l'opéron lactose, un gène de résistance à la kanamycine et au chloramphenicol. Ce plasmide est néanmoins mobilisable (présence d'une origine de transfert \textit{oriT}, et des gènes \textit{mobC} et \textit{mobB}).
Ce plasmide a été fourni par le docteur Xavier Charpentier du laboratoire de Microbiologie, Adaptation et Pathogénié de Lyon et a déjà été utilisé avec succès pour transformer des bactéries du genre Acinetobacter. La carte du plasmide est présentée en Figure 2.

\end{itemize}

% \begin{figure}[h]
%    \includegraphics[width=\textwidth]{images/pXDC116.png}
%    \caption{plop}
% \end{figure}

\section{Mise en place de l'élevage d'\textit{Aedes albopictus}}

\subsection{Mise en place de l'élevage}

L’élevage d'Ae. albopictus a été réalisé en insectarium de niveau de confinement I2 en partenariat avec la Rovaltain Research Company situé à Alixan, à proximité de Valence. Pour débuter cet élevage, des œufs provenant d’une population d’élevage du CRVOI (voir paragraphe 2.1.1) de la génération G31 ont été mis en bassine contenant de l'eau déchlorée, à une température ambiante de 25°C (Figure). 
Après éclosion des œufs, les larves ont été nourries avec un mélange composé de 3/4 de nourriture pour poisson d'aquarium (TetraMin ®) et 1/4 de levure de bière (Biover®).
À l'émergence, les adultes ont été maintenus à 25°C ±1°C, une humidité relative de 80\,\% (±10\,\%) et une photopériode de XXX, avec une solution nutritive stérilisée d'eau glucosée à 8\,\%. Les femelles adultes ont ensuite été nourries une fois par semaine sur du sang de lapin (donner détails de la composition et l’origine) au moyen de nourrisseurs artificiels Hemotek (figure).
Les œufs pondus ont été récupérés régulièrement puis mis à éclore.

\section{Obtention d'une souche d'\textit{Acinetobacter calcoaceticus} marquée à la GFP}

\subsection{Électrotransformation}

L'expérience d'éléctrotransformation s'est déroulée en deux temps : la bactérie doit tout d'abord être rendue compétente.
Pour cela, après une pré-culture pendant la nuit, la cinétique de croissance de la bactérie est élablie par mesure de densité optique (DO) à intervalle régulier.
Une fois cette cinétique déterminée, la culture est arrêtée en phase exponentielle de croissance, à une DO située entre 0,4 et 0,6.
Les bactéries subissent ensuite un choc thermique (incubation 15 minutes sur de la glace, puis lavage des cellules avec du tampon à 4°C), les rendant compétantes à la tranformation.
Les cellules compétentes sont ensuite concervées dans du glycérol.
Enfin, ces cellules sont soumises à un choc éléctrique dans l'éléctroporateur (Electroporator 2510), en présence de 75\,ng d'ADN plasmidique (le plasmide pHM2-GFP est au préalable extrait grâce au kit QIAprep Spin Miniprep Kit, Qiagen).

Les cellules sont enfin mises en culture sur milieu de culture solide LBM, en présence de kanamycine, pour maintenir une pression de séléction en faveur des cellules transformantes, et d'IPTG pour induire l'expression de GFP.

\subsection{Vérifications post-transformation}

\subsubsection{PCRs}
La présence des plasmides dans les colonies hypothétiquement transformantes est vérifiée par PCR sur le gène de la GFP, avec les amorces PnptII1F-Not et TendR-Not (taille de fragment attendue : 1.1kb), et sur les gènes de résistance à la kanamycine et \textit{LacZ} avec les amorces LacR et KanF (taille attendue : 2kb).

\subsubsection{Profil plasmidique}
La présence des plasmides est aussi testée en effectuant un profil plasmidique, qui consiste à extraire les plasmides des colonies via un protocole d'isolement des petits plasmides, puis de faire migrer l'ADN obtenu dans un gel d'agarose à 0.8\,\% pendant 24h.

\subsubsection{Observation au microscope à épifluorescence}
Le phénotype induit par l'expression de GFP en présence de l'inducteur IPTG est vérifié par l'observation de colonies fraiches (mises en culture la veille et incubées sur milieu LBM solide en présence d'IPTG et de kanamycine) en microscopie à épifluorescence.
En présence de lumière de longueur d'onde de 475\,nm, les bactéries fluorescent en vert (foir figure).

\section{Infection par \textit{A. calcoaceticus}-GFP}

\subsection{Description des conditions}

Les moustiques seront séparés en cinq modalités, notés de C0 à C4.
La condition C0 est composée de moustiques dits <<sauvages>>, c'est-à-dire n'ayant subi aucun traitement.
La condition C1 est composée de moustiques ayant subi un traitement antibiotique comme décrit dans la section 2.4.2, puis infectés par la souche d'\textit{A. calcoaceticus} marquée à la GFP, comme décrit dans la partie 2.4.3.
La condition C2 est composée de moustiques ayant subi un traitement antibiotique comme décrit dan la section 2.4.2, sans infection avec la bactérie, mais recevant le même traitement antibiotique que la condition C1 au moment de l'infection de celle-ci (voir section 2.4.2).
La condition C3 est composée de moustiques n'ayant pas subi de traitement antibiotique avant l'infection, puis infectés par la souche d'\textit{A. calcoaceticus} marquée à la GFP, comme décrit dans la partie 2.4.3.

\subsection{Traitement antibiotique des moustiques}

\subsubsection{Antibiogramme de la souche \textit{A. calcoaceticus} et modalités de traitement antibiotique}

Afin de mettre en place un traitement antibiotique adapté, maximisant son impact sur les bactéries de genre \textit{Acinetobacter} et minimisant celui sur l'endosymbiote \textit{Wolbachia}, la souche d'\textit{A. calcoaceticus} utilisée ici a été soumis à un test d'antibiogramme, afin de déterminer ses sensibilités et résistances à un certain nombre d'antibiotiques.
L'antibiogramme a été effectué tout d'abord au moyen du système automatisé VITEK© commercialisé par Biomérieux (disponible à la plateforme PARMIC de l'UMR 5557), puis les antibiotiques séléctionnés sont testés sur culture liquide.
Pour les tests en milieu liquide, une gamme de dillution des antibiotiques Péniciline/Streptomycine et Tétracycline est effectuée, puis inoculée par la souche testée, à une densité optique donnée (0.62 à 600\,nm).
La densité optique est mesurée après 24h d'incubation à 30°C, pour déterminer la concentration minimale d'antibiotique ayant un impact sur la bactérie.

\subsubsection{Application du traitement et vérification du statut d'infection par PCR}

Le premier traitement antibiotique sur la génération 32 est effectuée en prennant en compte le fait que la concentration en cellules bactériennes est beaucoup moins importante dans le moustique (env 10⁴ cellules par moustique (source)), par rapport à celle utilisée pour faire le test décrit plus haut (de l'ordre de 10⁸ UFC/ml), et que le moustique absorbe en moyenne 2\,µl de jus sucré lors d'un repas sucré.
La concentration choisie pour ce premier traitement est donc de 0.05\,µg/µl de gentamycine, 0.25\,µg/µl de streptomycine et 0.15\,µg/µl de péniciline, administré dans le repas sucré des adultes G32.

Suite à ce traitement, la présence d'\textit{Acinetobacter} est vérifiée par PCR nichée (voir partie 2.5).
Ce traitement ne semblant pas efficace au vu de ce test (voir partie résultats), un autre traitement est effectué sur les larves de la génération suivante.

Le traitement des larves se déroule sur le stade L4, à savoir le dernier stade avant la métamorphose en nymphe, sui émergera ensuite en adulte.
Les larves sont regroupées par lots de 30 individus dans des tubes Falcon de 50\,ml, dans 30\,ml de solution d'eau autoclavée et d'antibiotiques (gentamycine : 15\,µg/ml; péniciline/streptomycine respectivement 10 unités et 10\,µg/ml).
Ces tubes sont laissés ouverts dans des cages lavées à l'alcool, pour permettre l'émergence des adultes.
L'efficacité du traitement est aussi vérifié par PCR nichée (voir partie 2.5).


\subsection{Infection des moustiques par la bactérie marquée}

Les modalités d'infection des moustiques sont inspirées des travaux de \textcite{bahia2014}.
La bactérie, après une pré-culture de 14\,h puis un lavage dans une solution de NaCl à 0.8\,\%, est incorporée au repas sucré des cages correspondant aux modalités infectées, accompagné de kanamycine afin de fournir une pression de séléction en faveur de la souche comportant le plasmice comportant le gène de résistance à la kanamycine.
La solution nutritive donnée à ces adultes sera donc composée de 10⁸\,UFC d'\textit{Acinetobacter}-GFP, 100\,µg/ml de kanamycine et 8\,\% de saccharose.

Des témoins, comme décrit dans la partie 2.4.1, seront nouris de solutions ne comportant soit que la bactérie, soit que la kanamycine.

\subsection{Conception et mise en \oe{}uvre des expériences de mesure de traits d'histoire de vie}

% En se basant sur les conseils de l'entomologiste \textsc{Raharimalala} Fara Nantenaina et les documents délivrés par l'ECDC\footnote{European Center for Disease Prevention and Control} \cite{ecdc}, nous avons planifié de mesurer divers traits d'histoire de vie\footnote{L'histoire de vie est la distribution des évènements importants au cours de la vie d'un individu qui contribuent directement à la production et à la survie des descendants. Elle est donc une notion centrale en biologie évolutive} comme suit :

\paragraph{Survie des adultes.} Ce trait est mesuré sur des moustiques du même âge (±\,2 jours), isolés dans des petites cages (17.5\,×\,17,5\,×\,17.5\,cm). La date et le sexe des morts seront notés au fur et à mesure du déroulement de l'expérience, et stockés dans une base de données pour des analyses statistiques ultérieures.
En raison de la forte mortalité subie par les moustiques ayant subi le traitement antibiotique (voir section 2.4), et du faible nmbre d'individus adultes qui en résulte, il n'a pas été possible d'isoler des cages de 30 individus de ces conditions.
Ce suivi n'a pu être effectué que sur les individus témoins.

\paragraph{Taille de la ponte.} Ce trait est mesuré en isolant des femelles juste après leur premier gorgement, dans des tubes Falcon de 50\,ml préparés pour la ponte (figure 3A). Le nombre d'\oe{}ufs par femelle est compté 4 jours après l'isolement.
Un autre trait, la \textbf{fécondité}, est mesuré à partir de ces \oe{}fs. Il s'agit du rapport [Nombre d'\oe{}ufs par femelle/Nombre de larves], déterminé suite au dénombrement des larves une semaine après le comptage des \oe{}ufs.

\paragraph{Durée du cycle gonotrophique.} Il s'agit du temps séparant le premier gorgement et la première ponte d'une femelle. Ce trait est mesuré sur les femelles isolées du paragraphe précédent. La présence ou non d'\oe{}ufs est vérifiée chaque jour qui suit le gorgement.

\section{Suivi post-infection de \textit{Acinetobacter calcoaceticus}}

\subsection{Amplification du gène \textit{rrs}}

Les amplifications du gène codant la sous-unité ribosomique 16S des eubactéries ont été réalisées en utilisant 30\,ng d'ADN matrice et les amorces généralistes pA (5' AGAGTTTGATCCTGGCTCAG 3') et pH (5' AAGGAGGTGATCCAGCGCA 3') \cite{edwards1989}.
La réaction a été réalisée dans un volume final de 25\,µl contenant 20\,nM de chaque amorce, 40\,µM de dNTPs (Invitrogen), 0,035U de polymérase Expand, 1.5\,mM de MgCl\textsubscript{2} (Roche, Suise), 25\,µg/ml de protéine T4/32 (Roche, Suise), dans 1X de tampon de réaction (Invitrogen, France).
Le programme de PCR, effectué dans le thermocycleur T1-thermobloc (Biometra, France), débute par une dénaturation de 3\,min à 95°C suivie de 35 cycles comprenant 30\,sec à 94°C, 40\,sec à 55°C et 1\,min 30\,sec à 72°C, puis une élongation finale de 10\,min à 72°C.

\subsection{Amplification PCR 16S genre-spécifique}

Les PCRs 16S genre-spécifiques ont été réalisées en utilisant 1\,µl du produit PCR 16S généraliste dans un volume final de 25\,µl coprenant les amorces spécifiques du genre \textit{Acinetobacter} : Ac (5' GCGCCACTAAAGCCTCAAAGGCC 3') \cite{kenzaka1998} et Acin1 (5' ACTTTAAGCGAGGAGGAGGCT 3') \cite{sanguin2006}.
Les amorces ont été utilisées à une concentration de 0,2\,µM avec 40\,µM de dNTPs (Invitrogen, France), 0,5\,U de Taq polymérase (Invitrogen, France), 2\,mM de MgCl\textsubscript{2}, dans un tampon 1X (Invitrogen, France).
La réaction a été réalisée selon le programme suivant : 5\,min de dénaturation à 95°C, 35 cycles d'amplification de 1\,min à 94°C, 1\,min à 58°C, 1 minute à 72°C, suivie d'une élongation finale de 10\,min à 72°C, dans le thermocycleur C1000 Thermal Cycler (Biorad, France).
\chapter{Matériel et méthodes}

\section{Matériel biologique}

\paragraph{Moustiques.}
\label{par:moustiques}
La population de moustiques \esp{Aedes albopictus} qui est utilisée dans cette étude est issue d'une lignée d'élevage de laboratoire, fournie par l'Institut Pasteur, et provenant initialement de l'île de la Réunion\footnote{[Ref. néc.]}.
Le stage débute avec la génération 31 de cette ligée de moustiques, comptée depuis sa capture dans le milieu naturel.

\paragraph{Souches bactériennes.}
%Le choix de l'espèce \esp{Acinetobacter calcoaceticus} a été motivé par sa forte prévalence mise en évidence lors d'études précédentes [Publi Guillaume], ainsi que par sa cultivabilité et les capacités de transformation décrites dans de nombreuses souches de ce genre bactérien [Ref.].
La souche bactérienne d'\esp{Acinetobacter calcoaceticus} choisie comme modèle pour ces expériences est issue d'un isolement provenant de la flore d'un moustique capturé sur l'Île de la Réunion, en [Année, publi]%\cite{zouache2011}.
% Pourquoi la souche D ?

\paragraph{Plasmides}
[Ici, description des plasmides (MAP et italiens), ref des italiens (et MAP), illustration par carte plasmidique].

% \begin{figure}[h]
%    \includegraphics[width=\textwidth]{images/pXDC116.png}
%    \caption{plop}
% \end{figure}

\section{Mise en place de l'élevage d'\esp{Aedes albopictus} et ajustement des mesures de THV}

\subsection{Mise en place de l'élevage}

L'élevage et les mesures sur moustique vivant sont effectués en insectarium de niveau de confinement n°2 (conformément à la Directive 2008/61/CE), grâce à un partenariat avec la Rovaltain Research Company (anciennement Pôle Écotox), à Valence.

\subsection{Conception et mise en \oe{}uvre des expériences de mesure de traits d'histoire de vie}

\section{Obtention d'une souche d'\esp{Acinetobacter calcoaceticus} marquée à la GFP}

\subsection{Électrotransformation}

\subsection{Vérifications post-transformation}

\paragraph{PCRs}
\paragraph{Profil plasmidique}
\paragraph{Repiquages successifs}
\paragraph{Observation au microscope à épifluorescence}

\section{Traitement antibiotique des moustiques}

\subsection{Antibiogramme de la souche A. calcoaceticus et modalités de traitement antibiotique}

\subsection{Application du traitement}

\section{Infection des moustique par \esp{Acinetobacter calcoaceticus}-GFP}

\section{Suivi post-infection de \esp{Acinetobacter calcoaceticus}}

[PCR nichée pA/pH, Acineto]
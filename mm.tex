\chapter{Matériel et méthodes}

\section{Matériel biologique et génétique}

\subsection{Moustiques \textit{Aedes albopictus}}
\label{par:moustiques}
La population de moustiques \esp{Aedes albopictus} provient d'une lignée d'élevage de laboratoire, hébergée par le Centre de Recherche et de Veille sur les Maladies Emergentes dans l'Océan Indien (CRVOI), à l'île de la Réunion.
Les moustiques originels ont été capturés en 2012 dans trois localités distinctes de l'Île de la Réunion (Saint Benoît, Sainte Suzanne et Saint Denis) afin de minimiser les problèmes de consanguinité.

\subsection{Isolat bactérien}
%Le choix de l'espèce \esp{Acinetobacter calcoaceticus} a été motivé par sa forte prévalence mise en évidence lors d'études précédentes [Publi Guillaume], ainsi que par sa cultivabilité et les capacités de transformation décrites dans de nombreuses souches de ce genre bactérien [Ref.].
La souche d'\esp{Acinetobacter calcoaceticus} a été obtenue après isolement sur milieu riche de culture à partir d'un broyat de moustique Ae. albopictus mâle. Ce dernier était issu d'une population de laboratoire dont la population originelle avait été capturée sur l'Île de la Réunion en 2006 (\textcite{zouache2009})
Le séquençage du gène \textit{rrs} codant pour l'ARN ribosomique 16S a montré 99\% d'identité avec l'espèce A. calcoaceticus. L'isolat a été déposé dans GenBank sous le numéro d'accession FJ688379 et l'identifiant KZ-OAlM.
% Pourquoi la souche D ?

\subsection{Plasmides}

Deux plasmides ont été utilisés lors des essais de transformation de l'isolat d'A. calcoaceticus en vue de son suivi \textit{in insecta}~:
\begin{itemize}
\item \textbf{Le plasmide pHM2-GFP} est non mobilisable (absence de gènes de l'opéron \textit{tra}), et possède une origine de réplication (\textit{Rep}) généraliste lui permettant de se répliquer dans de nombreux groupes bactériens.
Outre le gène marqueur \textit{GFP} (Green Fluorescent Protein), cloné sur le site NotI, le plasmide conprend, entre autres, un autre marqueur~: l'opéron lactose, induisant une coloration bleue des colonies en présence de X-Gal (ou 5-bromo-4-chloro-3-indolyl-beta-D-\\galactopyranoside, substrat chromogénique de la β-galactosidase), et un gène de résistance à la Kanamycine.
Le plasmide a été fourni par le docteur Guido Favia du Département de Médecine Expérimentale et de Santé Publique de Camerino en Italie, et a déjà été utilisé avec succès pour transformer les bactéries du genre Asaia chez le moustique Anopheles \cite{favia2007}
La carte du plasmide est présentée en Figure 1A.

\item \textbf{Le plasmide pXDC116} contient, outre le gène marqueur de la \textit{GFP} et l'opéron lactose, des gènes de résistance à la kanamycine et au chloramphenicol. Ce plasmide est néanmoins mobilisable (présence d'une origine de transfert \textit{oriT}, et des gènes \textit{mobC} et \textit{mobB}).
L'expression du gène \textit{GFP} est contrôlée par un promoteur issu de l'opéron lactose, induit par l'IPTG (isopropyl β-D-1-thiogalactopyranoside), un réactif analogue à l'allolactose, qui active la transcription des gènes placés sous le contrôle de ce promoteur.
Ce plasmide a été fourni par le docteur Xavier Charpentier du laboratoire de Microbiologie, Adaptation et Pathogénié de Lyon et a déjà été utilisé avec succès pour transformer des bactéries du genre Acinetobacter. La carte du plasmide est présentée en Figure 1B.

\end{itemize}

% \begin{figure}[h]
%    \includegraphics[width=\textwidth]{images/pXDC116.png}
%    \caption{plop}
% \end{figure}

\section{Mise en place de l'élevage d'\textit{Aedes albopictus}}

\subsection{Mise en place de l'élevage}

L'élevage d'\textit{Ae. albopictus} a été réalisé en insectarium de niveau de confinement I2 en partenariat avec la Rovaltain Research Company situé à Alixan, à proximité de Valence\footnote{http://pole-ecotox.com/}. Pour débuter cet élevage, des œufs provenant d'une population d'élevage du CRVOI (voir paragraphe 2.1.1) de la génération G31 ont été mis en bassine contenant de l'eau déchlorée, à une température ambiante de 25°C (Figure 2). 
Après éclosion des œufs, les larves ont été nourries avec un mélange composé de 3/4 de nourriture pour poisson d'aquarium (TetraMin ®) et 1/4 de levure de bière (Biover®).
À l'émergence, les adultes ont été maintenus à 25°C ±1°C, une humidité relative de 80\,\% (±10\,\%) et une photopériode de 12:12, avec une solution nutritive stérilisée d'eau glucosée à 8\,\%. Les femelles adultes ont ensuite été nourries une fois par semaine au moyen de nourrisseurs artificiels Hemotek (figure 3) avec du sang de lapin stérile conservé dans du tampon citrate 12.5\,\%, livré dans des alicots de 10\,ml par la société BCL (Bergerie de la Combe aux Loups).
Les œufs pondus ont été récupérés régulièrement puis mis à éclore.

\section{Obtention d'une souche d'\textit{Acinetobacter calcoaceticus} marquée à la GFP}

\subsection{Électrotransformation}

Dans un premier temps, l'ADN plasmidique (pHM2-GFP ou pXDC116) contenu dans des cellules d'E. coli a été extrait selon la procédure décrite dans le kit QIAprep Spin Miniprep Kit (Qiagen, Pays ?).

L'isolat d'\textit{A. calcoaceticus} conservé en glycérol à -80°C a été remis en culture sur gélose LBm (Luria-Bertani modifié) et incubé 24h à 30°C.
Une pré-culture a été réalisée en inoculant 5\,ml de milieu LBm liquide par une colonie unique.
Après incubation pendant 12\,h à 30°C sous agitation, la pré-culture a été inoculée à 2×25\,ml de milieu LBm et la culture incubée à 30°C.
À intervalle de temps régulier, une mesure de densité optique a été réalisée au spectrophotomètre Safas® UV-mc\textsuperscript{2} jusqu'à atteindre une DO\textsubscript{600\,nm} comprise entre 0,4 et 0,6.
Parallèlement, un dénombrement par étalement sur géloses LBm de 100\,µl des dilutions 10\textsuperscript{5} à 10\textsuperscript{8} de la culture à la DO choisie a été effectué afin d'évaluer \textit{a posteriori} le nombre de bactéries compétentes.
Pour les rendre compétentes à la transformation, les cellules ont ensuite subi un choc thermique de 15 min dans la glace puis centrifugées 10 min à 6000 rpm à 4°C. 
Les culots ont été rincés à deux reprises dans 25\,ml d'eau puis repris dans 200\,µL final de glycérol 10\,\% froid.
Les cellules ont été conservées à -80°C à raison de 50\,µl de cellules compétentes par tube.

50\,µl de cellules compétentes (10\textsuperscript{12}\,CFU/ml) sont ensuite placées dans une cuve d'éléctroporation de 0,2\,cm de diamètre, puis soumises à un choc éléctrique de 2\,500\,V dans l'éléctroporateur (Electroporator 2510) en présence de 75ng d'ADN plasmidique.
Les cellules sont ensuite incubées à 30°C pendant 4h, pour que le plasmide puisse exprimer les gènes de résistance en vue de la séléction.

Enfin, les transformants ont été sélectionnés par culture sur géloses LBM, en présence de 50\,µg/ml de kanamycine et 0,5\,mM d'IPTG pour les cellules transformées avec pXDC116, ou 100\,µg/ml de kanamycine et la même quantité d'IPTG pour les cellules transformées avec pHM2-GFP.

\subsection{Analyse des transformants}

Différentes méthodes ont été utilisées pour vérifier la nature des colonies obtenues après transformation~:
\paragraph{Une PCR sur colonie sur le gène marqueur de la GFP.}
L'amplification a été réalisée dans un volume final de 25µl comprenant les amorces spécifiques PnptII1F-Not et TendR-Not (voir tableau)
(g-ne \textit{GFP}, taille attendue 1.1\,kb) ou LacR et KanF (plasmide pHM2, taille attendue 2\,kb) à une concentration de 0,\,μM avec 0,04\,µM de dNTPS (Invitrogen, France), 0,02\,U de Taq polymérase (Invitrogen, France), 1,63\,mM MgCl\textsubscript{2} dans un tampon 1X (Invitrogen, France).
La PCR a été réalisée selon le programme suivant: 3\,min de dénaturation à 95°C, 35 cycles d'amplification de 1\ min 95°C, 1\,min 60°C, 2\,min 72°C, suivie d'une élongation finale de 10 min à 72°C.
L'ensemble des PCR a été effectué dans le thermocycleur T1-thermobloc (Biometra, France).
L'ADN obtenu à partir de l'extraction plasmidique a servi de témoin positif.

\paragraph{Un profil plasmidique.}
Parallellement à l'analyse par PCR, la présence des plasmides a été vérifiée par une méthode de profl plasmidique mise au point par \textcite{seifert1994}, déjà éprouvée sur les isolats d'\textit{Acinetobacter} sp. provenant du moustique par \textcite{minard2013}.
Cette méthode consiste à comparer le profil de migration (nombre et taille) des plasmides extraits des isolats obtenus lors de l'électrotransformation, avec les témoins suivants~: Le témoin positif est le plasmide pur utilisé pour l'électro-transformation, et le témoin négatif est constitué par la souche d'\textit{Acinetobacter calcoaceticus} utilisé dans l'éléctro-transformation, mais n'ayant pas subi celle-ci.

Pour cela, les souches ont été isolées sur milieu LBm solide puis mises à incubation à 30°C pendant 24\,h.
Un quart des colonies obtenues sont inoculées dans 1\,ml de tampon (2,5\,M NaCl; 10\,mM EDTA), puis centrifugées à 13,2\,rpm.
Le culot ainsi obtenu est ensuite lysé d'abord par l'action d'un lysozyme (1,8\,mg/ml), puis par lyse chimique avec le détergent triton.
Le lysat ainsi obtenu est ensuite centrifugé à 33\,000\,g dans l'ultracentrifugeuse Optima™ Max-XP (Beckman Coulter, USA), puis lavé avec du tampon CTAB.
Enfin, les protéines sont extraites avec un mélange de phénol, chlorophorme, et alcool isoamylique (25:24:1), et l'ADN précipité à l'isopropanol, centrifugé puis repris dans de l'eau distillée.

Enfin, les échantillons sont migrés dans du gel d'agarose à 0,8\,\%, puis révélés par un bain de bromure d'éthidium.

\paragraph{Observation au microscope à fluorescence}
Le phénotype induit par l'expression de GFP en présence de l'inducteur IPTG est vérifié par observation des colonies au microscope à épifluorescence (AxioImager Z1). En présence de lumière de longueur d'onde de 475\,nm, les bactéries fluorescent en vert.

\section{Infection par \textit{A. calcoaceticus}-GFP}

\subsection{Description des modalités d'étude}

Quatre modalités d'infection des moustiques ont été réalisées: des moustiques n'ayant subi aucun traitement (condition C0), des moustiques traités aux antibiotiques puis infectés (condition C1) ou non (condition C2) par la souche \textit{A. calcoaceticus}-GFP et enfin des moustiques directement infectés par la souche \textit{A. calcoaceticus}-GFP (condition C3).

\subsection{Traitement antibiotique des moustiques}

\subsubsection{Antibiogramme de la souche \textit{A. calcoaceticus}}

Le traitement antibiotique a pour but d'éliminer la flore résidente d'\textit{Acinetobacter} chez les moustiques, au préalable de l'infection par la bactérie marquée.
Un critère important conditonnant le choix des antibiotiques à utiliser est leur impact modéré voire nul sur l'endosymbiote \textit{Wolbachia}.
Pour cela, un antibiogramme realisé à l'aide du système automatisé Vitek2® (Biomérieux, France) a été préalablement effectué sur la souche \textit{A. calcoaceticus} afin de tester la sensibilité de la bactérie vis-à-vis d'un panel d'antibiotiques.
Pour cela, une des colonies isolées d'une culture pure de la bactérie est mise en suspention dans une solution de NaCl à 0,45\,\%.
cette solution de bactéries est ensuite inoculée dans un tube de solution de NaCl, de manière à obtenir une solution de disturbance entre 0,5 et 0,6\,\%.
Le tube ainsi obtenu est placé dans une cassette VK2C3285 (Biomérieux, France) contenant la batterie de tests antibiotiques, puis celle-ci est placée dans la machine Vitek2®.

À l'issue du test, les antibiotiques sélectionnés ont été testés en milieu liquide afin de déterminer la concentration minimale inhibitrice (CMI) de chaque antibiotique sur la bactérie.
Pour cela, une suspension de bactéries dans du LBm est préparée pour correspondre à une DO\textsuperscript{600} de 0,60, et est aliquotée dans des tubes.
Ces suspensions sont ensuite confrontées à des concentrations croissantes d'antibiotiques, afin d'établir une gamme de dillution de ces antibiotiques.
La croissance ou non de ces cultures est mesurée par densité optique après 24\,h d'incubation à 30°C sous agitation.
Une DO\textsuperscript{600} suppérieure à 0,60 après cette incuvation est synonyme de croissance. En revanche, les concentrations d'antibiotiques provoquant une DO\textsuperscript{600} restée à 0,60 après incubation sont considérés comme suppérieures à la CMI.

\subsubsection{Modalité d'application du traitement antibiotique sur les moustiques}

Le traitement antibiotique a été appliqué sur des moustiques adultes de la G32 ainsi que sur des larves de la G33.

Pour déterminer les concentrations optimales d'antibiotiques à administrer aux moustiques adultes, deux critères d'évaluation ont été pris en compte :
(i) la densité bactérienne présente dans un moustique (~10⁴ cellules selon source\footnote{Note temp : Trouver une source pour ça.
On a décidé de cette valeur lors d'une discussion avec Patrick, mais ona  jamais vraiment sourcé cela}) est inférieure d'un facteur 10\textsuperscript{4} à la concentration utilisée lors des tests antibiogrammes réalisés précédemment (~10\textsuperscript{8} UFC/ml), et (ii) la quantité de jus sucré consommée par un moustique est de l'ordre de 2µl.
Compte tenu de ces observations, nous avons choisi d'appliquer un traitement 10\,000 fois moins concentré de chaque antibiotique aux moustiques, Les antibiotiques ont été administrés lors du repas sucré, pendant 3 jours consécutifs. 

Le traitement antibiotique des larves a été réalisé sur les stades L4.
Pour cela, des lots de 30 individus ont été constitués dans des tubes Falcon de 50\,ml contenant 30\,ml d'eau déchlorée autoclavée supplémentée avec la solution d'antibiotiques précédemment déterminée mais concentrée 100\,000 fois, comme utilisées dans ce précédentes études \cite{dong2009}.
Les tubes ont été positionnés dans des cages préalablement nettoyés à l'alcool pour permettre l'émergence des adultes.

\subsubsection{Amplification PCR nichée \textit{Acinetobacter} sp.}

L'efficacité des traitements antibiotiques a été vérifiée par PCR nichée.
Au préalable, l'ADN des moustiques a été extrait selon le protocole décrit dans \textcite{zouache2009}.
Les moustique sont tout d'abord rincés alternativement avec de l'eau ultra-pure et de l'alcool à 70\,\%, afin d'éliminer le microbiote présent à la surface de l'insecte, pour n'extraire que celui ayant colonisé l'intérieur de celui-ci.
Ils sont ensuite placés dans des tubes de 2\,ml en présence d'une bille en acier inoxidable, plongés dans de l'azote liquide pendant 10\,sec afin de les rendre cassants et d'inactiver les DNases, puis broyés au moyen du broyeur Retsch Vibration mill MM 2000 (Retsch, Allemagne).
Suite au broyage mécanique, un broyage chimique est appliqué au moyen d'une solution de lyse composée du détergent CTAB et de β-mercaptoéthanol, à 60°C, suivi d'un traitement à la RNase, à 37°C.
L'ADN est ensuite séparé des protéines par ajout de phénol/chlorophorme/alcool isoamylique (25:24:1), agitation puis centrifugation.
La phase acqueuse contenant l'ADN est ensuite purifiée par ajout de chloroforme/alcool isoamylique (24:1), agitation puis centrigugation.
Enfin, l'ADN contenu dans la phase acqueuse est récupéré, puis précipité par ajout d'ispropanol, au froid pendant 24h.
Après centrifugation, le culot est récupéré, lavé à l'alcool 75\,\%, puis remis en suspension dans 15\,µl de tampon TE (Tris 10\,mM; EDTA 1\,mM)

La 1ère amplification ciblant le gène codant la sous-unité ribosomique 16S des eubactéries a été réalisée en utilisant 30\,ng d'ADN matrice et les amorces généralistes pA et pH (Tableau XXX).
La réaction a été réalisée dans un volume final de 25\,µl contenant 20\,nM de chaque amorce, 40\,µM de dNTPs (Invitrogen), 0,03\,U de polymérase Expand, 1.5\,mM de MgCl\textsubscript{2} (Roche, Suise), 25\,µg/ml de protéine T4/32 (Roche, Suise), dans 1X de tampon de réaction (Invitrogen, France).
Le programme de PCR, effectué dans le thermocycleur T1-thermobloc (Biometra, France), débute par une dénaturation de 3\,min à 95°C suivie de 35 cycles comprenant 30\,sec à 94°C, 40\,sec à 55°C et 1\,min 30\,sec à 72°C, puis une élongation finale de 10min à 72°C.

La seconde PCR ciblant une région spécifique du gène 16S d'\textit{Acinetobacter} a été réalisée en utilisant 1µl du produit PCR 16S généraliste dans un volume final de 25µl comprenant les amorces spécifiques du genre Acinetobacter~: Ac et Acin1 (Tableau 1).
Les amorces ont été utilisées à une concentration de 0,2\,µM avec 40\,µM de dNTPs (Invitrogen, France), 0,5\,U de Taq polymérase (Invitrogen, France), 2\,mM de MgCl\textsubscript{2}, dans un tampon 1X (Invitrogen, France).
La réaction a été réalisée selon le programme suivant~: 5\,min de dénaturation à 95°C, 35 cycles d'amplification de 1\,min à 94°C, 1\,min à 58°C, 1 minute à 72°C, suivie d'une élongation finale de 10\,min à 72°C, dans le thermocycleur C1000 Thermal Cycler (Biorad, France).

Les produits PCR sont ensuite migrés dans un gel d'agarose 1\,\%, pendant 20 minutes, dans un champs éléctrique de 100\,V, en présence de BET.
Les gels sont lus par photographie UV, par l'appareil Gel Doc 2000 (Biorad, France).

\subsection{Infection des moustiques par la bactérie marquée}

La veille de l’infection, 10 pré-cultures de la bactérie transformée ont été réalisées dans 5\,ml de LBm supplémenté avec 50\,µg/ml de Kanamycine et 0,5\,mM d'IPTG.
Après 14h d'incubation, la DO\textsubscript{600\,nm} de chaque culture a été mesurée afin d'estimer la concentration bactérienne.
Les cultures ont été centrifugées à 6\,000\,rpm pendant 10\,min et les culots rincés dans une solution stérile de NaCl à 0.8\%.
Les culots ont été repris dans 800\,ml de solution sucrée 8\,\% préalablement filtrée sur un filtre à 0,20\,µm
%(décrire les 2 étapes consécutives de préparation svp),
% supplémentée ou non avec de la kanamycine à 50 µg/ml.
Cette solution sucrée est ensuite aliquoté dans des bouteilles de 200\,ml, et supplémentée par les éléments suivants, formant quatre modalités :
(i) pas d'ajout; (ii) \textit{Acinetobacter}-GFP + kanamycine; (iii) \textit{Acinetobacter}-GFP seul; (iv) Kanamycine seule.
%Décrire la procédure de préparation des cotons imbibés avec la solution infectée.
% Parallèlement, des nourrissoirs contenant du jus sucré uniquement ou supplémenté avec de la kanamycine (50\,µg/ml) ont également été préparés. L’ensemble des nourrissoirs a été déposé dans les cages selon les modalités décrites dans le paragraphe 2.4.1.
La kanamycine est ajoutée à 50\,µg/ml, et la bactérie à 10\textsuperscript{8}\,UFC/µl.
Les nourrissoirs (figure 2C) sont fabriqués sous hotte PSM, avec des béchers de 25\,ml autoclavés, et du coton autoclavé.
Les nourrissoirs sont ensuite placés dans les cages de leurs conditions correspondantes, pendant 24h, puis remplacés par des nourrissoirs contenant du jus sucré simple, stérile.

% Il faut quelque part que tu donnes aussi le nombre de moustiques / cages.
% Les temps d’étude/prélèvement des individus et leur « destination »
% → Figure 4

La conception du plan expérimental tenant compte de l’ensemble des modalités d’étude est présentée dans la Figure 4.

\section{Suivi post-infection de \textit{Acinetobacter calcoaceticus}}

\subsection{Amplification du gène \textit{rrs}}

Les amplifications du gène codant la sous-unité ribosomique 16S des eubactéries ont été réalisées en utilisant 30\,ng d'ADN matrice et les amorces généralistes pA (5' AGAGTTTGATCCTGGCTCAG 3') et pH (5' AAGGAGGTGATCCAGCGCA 3') \cite{edwards1989}.
La réaction a été réalisée dans un volume final de 25\,µl contenant 20\,nM de chaque amorce, 40\,µM de dNTPs (Invitrogen), 0,035U de polymérase Expand, 1.5\,mM de MgCl\textsubscript{2} (Roche, Suise), 25\,µg/ml de protéine T4/32 (Roche, Suise), dans 1X de tampon de réaction (Invitrogen, France).
Le programme de PCR, effectué dans le thermocycleur T1-thermobloc (Biometra, France), débute par une dénaturation de 3\,min à 95°C suivie de 35 cycles comprenant 30\,sec à 94°C, 40\,sec à 55°C et 1\,min 30\,sec à 72°C, puis une élongation finale de 10\,min à 72°C.

\subsection{Amplification PCR 16S genre-spécifique}

Les PCRs 16S genre-spécifiques ont été réalisées en utilisant 1\,µl du produit PCR 16S généraliste dans un volume final de 25\,µl coprenant les amorces spécifiques du genre \textit{Acinetobacter} : Ac (5' GCGCCACTAAAGCCTCAAAGGCC 3') \cite{kenzaka1998} et Acin1 (5' ACTTTAAGCGAGGAGGAGGCT 3') \cite{sanguin2006}.
Les amorces ont été utilisées à une concentration de 0,2\,µM avec 40\,µM de dNTPs (Invitrogen, France), 0,5\,U de Taq polymérase (Invitrogen, France), 2\,mM de MgCl\textsubscript{2}, dans un tampon 1X (Invitrogen, France).
La réaction a été réalisée selon le programme suivant : 5\,min de dénaturation à 95°C, 35 cycles d'amplification de 1\,min à 94°C, 1\,min à 58°C, 1 minute à 72°C, suivie d'une élongation finale de 10\,min à 72°C, dans le thermocycleur C1000 Thermal Cycler (Biorad, France).